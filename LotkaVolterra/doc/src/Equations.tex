%
% Author(s):
%       Takahiro Hashimoto
%

%% preamble %%
    \documentclass[a4j]{article}
    \hoffset -15 pt %1
    \voffset -40 pt %2
    \oddsidemargin 0 pt %3
    \topmargin 4 pt %4
    \headheight 20 pt %5
    \headsep 18pt %6
    \textheight 717 pt %7
    \textwidth 483 pt %8
    \marginparsep 18 pt %9
    \marginparwidth 18 pt %10
    \footskip 28 pt %11

    % include packages for equation
    \usepackage{amsmath}
    \usepackage{amssymb}
    \usepackage{bm}

    % include packages for header and fooder
    \usepackage{fancyhdr}
    \usepackage{extramarks}
    \pagestyle{fancy}
    \lhead{}
    %\rhead{\firstleftmark}
    \rhead{}
    \cfoot{}

    \bibliographystyle{ieeetran}


\begin{document}

    Lotka-Volterra Equations\cite{wikilv, wollv} are couples of ordinary differential equations as follows

    \begin{equation}
        \label{Eq:LV1}
        \frac{dp_1}{dt} = a p_1 -b p_1 p_2, 
    \end{equation}

    \begin{equation}
        \label{Eq:LV2}
        \frac{dp_2}{dt} = -c p_2 + d p_1 p_2, 
    \end{equation}

    where $p_1, p_2 \in \mathbb{R}^+ $ are dependent valiables of $t$, and $a, b, c, d \in \mathbb{R}^+ $ are constant parameters. Eliminating $t$ from Eqs.(\ref{Eq:LV1}) and (\ref{Eq:LV2}), conserved quantity of the system is derived as 
    \begin{equation}
        \label{Eq:cnsv}
        V = -a \log(p_2) + b p_2 -c \log(p_1) + d p_1.
    \end{equation}

    Solving Eq.(\ref{Eq:cnsv}) for $p_2$,  
    \begin{equation}
        p_2 = -\frac{a}{b} \:
            W \left( -\frac{b}{a} \exp \left(
            \frac{1}{a} \left(-c \log p_1 + d p_1 -V 
            \right) \right) \right),
    \end{equation}

    where $W$ is the Lambert W function\cite{wikilam}. 
    $W$ is multivalued function with two branches ($W_{-1}, W_{0}$). 
    We choose $W_0$ when $-b/a \, p_2 \leq -1$ and $W_{-1}$ for other case.

    \addcontentsline{toc}{section}{References}
    \bibliography{bibliography/Ref}


\end{document}
